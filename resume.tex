%
% LaTeX source of my resume
% =========================
%
% Created from a LaTeX template by Cies Breijs
%

% Start a document with the here given default font size and paper size.
\documentclass[10pt,a4paper]{article}

% Set the page margins.
\usepackage[a4paper,margin=0.75in]{geometry}

% Setup the language.
\usepackage[english]{babel} \hyphenation{Some-long-word}

% Makes resume-specific commands available.
\usepackage{resume}


\begin{document}  % begin the content of the document
\sloppy  % this to relax white-spacing in favor of straight margins

% title on top of the document
\maintitle{Nathaniel Waisbrot}{}{Last update on \today}

\nobreakvspace{0.3em}  % add some page break averse vertical spacing

% \noindent prevents paragraph's first lines from indenting
% \sbull is a spaced bullet
% \\ breaks the line into a new paragraph
\noindent\href{mailto:resume@waisbrot.net}{resume@waisbrot.net} \sbull
\textsmaller{p} 413-726-5098 \sbull
\href{https://www.linkedin.com/in/waisbrot}{www.linkedin.com/in/waisbrot}
\\
74 Standish St \#1, Cambridge, MA, 02138, USA

\spacedhrule{0.9em}{-0.4em}  % a horizontal line with some vertical spacing before and after

\roottitle{Work}

\headedsection
  {\textbf{\href{https://google.com}{Google}}}
  {\textsc{Mountain View, California}} {
    \\Devices and Services $\gg$ Fitbit
    \headedsubsection
      {Staff Software Engineer, SWE}
      {2022 -- 2023}
      {\bodytext{
        I was the senior-most individual contributor of the team decomposing Fitbit's legacy Java monolith as part of the effort to move it 
        into Google's internal production systems.

        I hosted a summer intern and coached her through creating a tool to automate the difficult process of
        releasing the monolith. The internship was a fun and positive experience for everyone and the team
        continues to use and improve the tool today.

        My team deals with a vast amount of technical debt that we need to triage. I added a requirement that
        all integration tests needed to pass before the code could be released. This required coordination
        with all the teams who own code in the monolith. Both before the change, to get as much agreement
        as possible, and after the change to help the many engineers who were unhappy that they could no longer
        release code (because tests were failing). I also worked with teams who owned the test code, because
        in some cases the tests themselves were the source of the bug.

        I made usability improvements to our release automation tool and changed our release process to be
        self-serve. This allowed many EU-based engineers to get their changes out faster, and also freed
        my team to do less operational toil and more impactful engineering work.
      }}
    \headedsubsection
      {Staff Software Engineer, SRE}
      {2021 -- 2022}
      {\bodytext{
        I helped the Fitbit organization adapt to new Google policies, such as the requirement that all user-facing
        changes receive Legal, Security, Product, and VP review before going live.

        When the Log4j vulnerability was announced, I coordinated Fitbit's response. This involved working with every
        code-owning team at Fitbit to determine whether or not they were vulnerable, building mitigation plans for
        teams who were, and providing daily status updates to the larger Google Security response team.
      }}
  }

\headedsection
  {\textbf{\href{https://fitbit.com}{Fitbit}}}
  {\textsc{San Francisco, California}} {
    \\Fitness wearables
    \headedsubsection
      {Staff Software Engineer, SRE}
      {2018 -- 2021}
      {\bodytext{
        I was the SRE in charge of onboarding a company that Fitbit acquired. I imported the test and production 
        infrastructure into Terraform (which allowed them to rapidly create and destroy additional test environments). 
        I helped the team to get alerts under control by working with Product to understand which had customer impact 
        and which could be safely deleted. I taught the team to manage and review incidents and when I left 
        they were a model for the rest of the company.

        As part of the main SRE rotation I was on-call for all major incidents and outages at Fitbit, handling problems 
        that spanned multiple teams and had PR/Legal risk. I kept responders' focus on mitigating the customer impact, 
        ran post-incident reviews, and gave briefings to executives when incidents ran longer than our goal (15 minutes 
        to detect, 1 hour to mitigate).

        As part of the main SRE team, I developed software solutions to improve reliability across the organization. 
        One of my projects was the Deployment Approval Service, which prevented deployment of services that didn't 
        meet SRE's base standards of reliability. This provided SWE with leverage to address technical debt by giving 
        an early and visible trade-off to the Product team.
      }}
  }

\headedsection  % sets the header for the section and includes any subsections
  {\textbf{\href{https://www.fireeye.com}{FireEye}}}
  {\textsc{Milpitas, California}} {%
  \\Cybersecurity
  \headedsubsection
    {Senior Staff Software Engineer}
    {2016 -- 2018}
    {\bodytext{
      I worked on FireEye Security Orchestrator, a tool for automating security operations. I contributed to all aspects 
      of the orchestration engine, but my primary focus was the API for Python-based plugins to communicate with the 
      Elixir-based engine. I built a Python framework to allow community developers to write Python code integrating 
      their security appliance with our orchestrator, and I wrote Elixir IPC code to communicate with those plugins.
      }}
    }

\headedsection  % sets the header for the section and includes any subsections
  {\textbf{\href{https://www.whoop.com}{WHOOP}}}
  {\textsc{Boston, Massachusetts}} {%
  \\A wearable and data analysis platform allowing elite athletes to optimize their training schedules.
  \headedsubsection
    {Team Lead}
    {2015 -- 2016}
    {\bodytext{
      Supervised two other full-time software engineers on the back-end team, working with the Product team to 
      prioritize and groom new features. 

      Managed a summer intern (planned a summer project, lead the hiring process, supervised the project) integrating multiple 
      external services (SalesForce, Shopify, GitHub, etc) with our own internal services.

      Architected and oversaw our transition to Kubernetes (from a hand-managed Ansible/Docker setup) and expanded our 
      use of PostgreSQL.
    }}

  \headedsubsection
    {Sr Software Engineer}
    {2014 -- 2015}
    {\bodytext{
      Architected and authored most of the server-side services and infrastructure. The WHOOP client apps send frequent 
      bursts of biometric data for near-real-time analysis and send and receive metadata describing users' daily activity 
      and performance. The server is composed of several in-house services (REST server, real-time data processing, slower 
      data processing, time-based actions, etc) and several AWS services (Kinesis, SQS, SNS, RDS PostgreSQL). My work was 
      primarily in Python, Erlang, and JavaScript.\\

      Some major technology choices I made: creating a suite of end-to-end tests, discarding Chef as a configuration 
      management system, introducing Docker, introducing Erlang, replacing OpenTSDB with PostgreSQL, introducing Amazon 
      Kinesis.
    }}
}

\headedsection
  {\textbf{\href{https://www.imprivata.com}{Imprivata}}}
  {\textsc{Lexington, Massachusetts}} {%
  \\Doctor-facing software for hospitals
  \headedsubsection
    {Sr Software Engineer}
    {2013 -- 2014}
    {\bodytext{
      Back-end engineer for Cortext, a HIPAA-compliant cloud-based messaging system. Designed and wrote the message alert 
      sub-system. Data-mined customer usage. Spearheaded the addition of regression tests, using a local VM to allow the 
      system to be tested with minimal changes to the code-base. Performed auxiliary ops tasks on a VPC of 50+ EC2 instances.
    }}
}

\headedsection
  {\textbf{\href{http://www.highfleet.com}{HIGHFLEET}}}
  {\textsc{Baltimore, Maryland}} {%
  \\Databases with an inference engine
  \headedsubsection
    {Software Engineer}
    {2010 -- 2013}
    {\bodytext{
      Worked on Highfleet's deductive database system: wrote builtin functions for the logic programming language, improved 
      communication with SQL-based back-ends, debugged query plans, helped write a new ANTLR-based language parser. 
      Part of a 4-person development team, extremely self- motivated; most weeks consisted of coming up with appropriate 
      enhancement or debugging tasks and then executing those.
    
      Primary technical contact for customers: wrote custom GUIs on demand for customers, wrote internal tools to support 
      ontologists, communicated extensively with customers. Sold a key customer a 50\% increase in their contract, then 
      produced software to fulfill the new requirements.
    }}
}

\headedsection
  {\textbf{Lockheed Martin} contracted to the \textbf{\href{https://www.arl.army.mil/www/default.cfm}{Army Research Laboratory}}}
  {\textsc{Adelphi, Maryland}} {%
  \headedsubsection
    {Software Engineer}
    {2009 -- 2010}
    {\bodytext{
      Integrated open-source scientific visualization and modeling software packages into a single installation for DoD supercomputer 
      users. Assisted ARL scientists in adapting their code to execute in parallel, for better performance on supercomputers. 
      Updated codes to run on new systems and architectures.
    
      Granted DoD Top Secret clearance and worked in a classified environment.
    }}
}

\spacedhrule{0.5em}{-0.4em}

\pagebreak[4]

\roottitle{Education}

\headedsection
  {\href{http://www.umd.edu}{University of Maryland}}
  {\textsc{College Park, MD}} {%
  \headedsubsection
    {Master of Science}
    {2008} 
    {\bodytext{Doctoral research in Natural Language Processing and Automated Planning}}
}

\headedsection
  {\href{https://www.vassar.edu}{Vassar College}}
  {\textsc{Poughkeepsie, NY}} {%
  \headedsubsection
    {Bachelor of Arts}
    {2002}
    {\bodytext{Concentration in Computer Science, with a minor in Comparative Religions}}
}

\spacedhrule{0.5em}{-0.4em}

\roottitle{Publicly-visible work}

\headedsection
  {\href{https://github.com/WhoopInc/dogstatsde}{A statsD client for DataDog}}
  {2016}
  {\bodytext{DataDog is a monitoring service that uses statsD with some enhancements. I wrote an Erlang client for their modifications for WHOOP to use in production code. The repository includes scripts to make Travis run tests and automatically publish releases to the Hex package repository.}}
\headedsection
  {\href{https://github.com/semiocast/pgsql/pull/24}{COPY support for pgsql}}
  {2015}
  {\bodytext{Pgsql is a PostgreSQL client library for Erlang. My patch adds support for PG's \texttt{COPY} command (bulk insert) and includes tests of the new feature and README changes to show people how to use it}}
\headedsection
  {\href{https://github.com/curl/curl/pull/351}{--proto-default option for Curl}}
  {2015}
  {\bodytext{Curl is a library and command-line tool for talking to URLs. I wanted it to use HTTPS by default, which was \href{http://stackoverflow.com/questions/31479263/can-curl-default-to-using-https}{not possible}, so I wrote a patch and worked with the Curl team to get it approved. My change was released as part of 7.45.0.}}
\headedsection
  {Pool-hall challenge for the \href{https://nethack.devnull.net}{/dev/null NetHack Tournament}}
  {2012}
  {\bodytext{NetHack is an old terminal-based multiplayer game. /dev/null hosts a yearly tournament with funny extra "challenges" patched into the game code. I wrote patch to the Tournament code adding a puzzle with boulders for pool-balls and holes in the floor for pockets. It's been used in the yearly tournament since 2012.}}

\spacedhrule{0.5em}{-0.4em}

\roottitle{Publications}

{\href{https://www.cs.umd.edu/~nau/papers/gerevini2008combining.pdf}{\textit{Combining Domain-Independent Planning and HTN Planning: The Duet Planner}}. Gerevini, Kuter, Nau, Saetti, and Waisbrot. European Conference on Artificial Intelligence (ECAI), 2008.\\
\\
\href{https://www.aaai.org/Papers/FLAIRS/2008/FLAIRS08-131.pdf}{\textit{Combining Heuristic Search with Hierarchical Task-Network Planning}}. Waisbrot and Kuter and K\"{o}nik. Florida Artificial Intelligence Research Society (FLAIRS), 2008.}

\spacedhrule{0.5em}{-0.4em}

\roottitle{Keywords}
  {
    Amazon Web Services, 
    Ansible, 
    Bash, 
    C, 
    Continuous Integration, 
    Docker, 
    Elixir, 
    Emacs, 
    Erlang, 
    Git, 
    Java,
    JavaScript, 
    Jenkins, 
    Microservices, 
    Perl, 
    PostgreSQL, 
    Python,
    Vagrant
  }


\end{document}

